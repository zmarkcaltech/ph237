\documentclass[11pt]{article}
%\setlength{\oddsidemargin}{0in}
%\setlength{\topmargin}{0.0in}
%\setlength{\textwidth}{6.7in}
%\setlength{\textheight}{8.5in}
%\usepackage{graphicx}
\usepackage{enumitem}
\usepackage{mathtools}
\usepackage[usenames,dvipsnames]{xcolor}

\usepackage{geometry}
 \geometry{
 letterpaper,
 textwidth=6.5in
 }
\usepackage[utf8]{inputenc}
%\usepackage{libertine}
%\usepackage{libertinust1math}
\usepackage[libertine,cmintegrals,cmbraces,vvarbb]{newtxmath}
%\usepackage{newtxmath}
%\usepackage[osf]{ebgaramond}
\usepackage[T1]{fontenc}
%\usepackage{palatino}
\usepackage{microtype}

\usepackage{fancyhdr}
\setlength{\headheight}{15pt}% ...at least 15
\usepackage[colorlinks=true]{hyperref}

\newcommand{\zach}[1]{\textcolor{ForestGreen}{#1}}

\pagestyle{fancy}
\fancyhf{}
\fancyhead[RE,LO]{\textcolor{BlueViolet}{Gravitational Waves}}
\fancyhead[LE,RO]{ph237: 2018}

\fancyfoot[RE,LO]{\textcolor{Orange}{Caltech}}
\fancyfoot[LE,RO]{\textcolor{Orange}{Physics, Math, and Astronomy}}

%\input mydefs.tex
\def\vev#1{\left\langle #1\right\rangle}
\def\hb{\hfill\break}

\begin{document}
%
\centerline{\large\bf  \hfill Assignment II \hfill  \today}

\medskip
\begin{description}
\item[{\bf Reading:}] Lecture Notes. Hartle NNN. \\
\item[{\bf Problems:} \hfill ] Due April 26, before class. \\
For these problems please pick one which is most interesting to you and complete it as far as you find interesting.
\end{description}


\medskip

\begin{enumerate}

\item
{\bf The precession of Mercury} \\
In this problem we numerically compute the precession of bound orbits in the Schwarzschild Geometry. This calculation explains the precession of Mercury's orbit.

\begin{itemize}

\item[\bf a)] Numerically evolve geodesic equations in the form derived in class (also in Ch. 9 of Hartle)
\begin{align}
\frac{1}{2}\left(\frac{dr}{d\tau}\right)^2+ V_{\rm eff}(r)&=\frac{e^2-1}{e} \nonumber \\
r^2\frac{d\phi}{d\tau}&=\ell
\end{align}
where $e$ is the energy per unit mass (including the mass energy) and $\ell$ is the angular momentum per unit mass. Use values of $e$ and $\ell$ that describe Mercury (since Mercury is moving at non-relativistic velocities, it is okay to use the non-relativistic expressions for $e$ and $\ell$ in terms of the orbital parameters $e$ and $a$, provided that you remember to include the mass energy).

\item[\bf b)] From your calculation of $r(\tau)$ calculate $\tau_r$, the length of proper time for $r$ to complete a full cycle. From your expression for $\phi(\tau)$, compute the precession of the orbit $\Delta \omega =\phi(\tau_r)-2\pi$.

\item[\bf c)] Compare your result to the measured precession of Mercury's orbit.

\end{itemize}

\item
{\bf Hulse-Taylor Binary and Gravitational Radiation} \\
In 1974, Hulse and Taylor observed a signal from a pulsar with rapidly changing period. They quickly identified this as being a pulsar in a relativistic binary. Subsequent analysis from Weisberg and Taylor verified that the orbital decay was consistent with general relativity to a high precision.

In this problem we will analyze the binary orbit, its precesion and decay in order to confirm these results.
\begin{itemize}

\item[\bf a)] Compute the orbital precesion of a binary with $r \gg G M$.

\item[\bf b)] Since the eccentric orbits in the Schwarzschild metric are not closed (as they are in Newtonian gravity), show that the orbit precesses, and that therefore, angle of the \emph{periastron} (point of closest approach) slowly varies with time. This is the calculation done by Einstein for the case of Mercury which gave him "palpations of the heart".

\item[\bf c)] Using the relevant numbers for the Hulse-Taylor binary (PSR 1913+16), compute the same thing. Include relativistic corrections?

\end{itemize}




\clearpage
\item
{\bf Phased Array of Radiators} \\
We would like to be able to generate a detectable amount of gravitational radiation on the earth so as to be able to test our detectors. Our approach will be to create a \emph{phased array} of rotating bars. By adjusting the phase of these rotating bars, we will be able to focus the energy into a narrow beam aimed at a detector on the other side of the earth.
\begin{itemize}

\item[\bf a)] Using the quadrupole formula and your results from hwk \#2, compute the gravitational radiation in the radiation zone for a single rotating bar.

\item[\bf b)] Using (e.g. python or Mathematica) plot the angular distribution of the radiation ($h(\theta, \phi)$). You will need to make 1 plot for each of the 2 polarizations. Recall that the radiation, as projected on your detector, will depend only on the distance and the projection of the reduced moment of inertia tensor onto the detector coordinate system.

\item[\bf c)] Now make a 2D array of $N \times N$ such bars. Compute and plot the angular distribution of the field.

\item[\bf d)] Adjusting the phase of each of the bars individually, maximize the radiation in the vertical direction (i.e. normal to the array). Note that you can either analytically compute the desired phase shift for each bar, or tune them all numerically (e.g. using some global optimization algorithm). What is the half-width of the directed beam from the array? How much of the energy is contained within this beam and how much is directed to side-lobes?

\end{itemize}



\end{enumerate}

\bigskip
{\color{Sepia} \hrule}
\end{document}
